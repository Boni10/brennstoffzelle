Mit immer größer werdender Kritik an fossilen Brennstoffen in Motorbetriebenen Fahrzeugen und Flugzeugen wird die Suche nach einem passenden Ersatz immer bedeutender. Nachdem sich nun auch das Elektroauto aufgrund seines $CO_2$ Abdrucks große Kritik holte, ist eine Andere Technologie notwendig. Eine Technologie, die Effizient, umweltschonend und in großen Massen verfügbar ist.
Über eine mögliche Alternative soll nun hier berichtet werden, die Brennstoff und Solarzelle.
Ottomotoren haben einen geringen Wirkungsgrad und nutzen den endlichen fossilen Brennstoff Erdöl.
Rund 15 \% der Erde bestehen aus Wasserstoff, der sich zu seiner elementaren Form auch leicht herstellen lässt.
Das größte bisherige Problem ist, dass eine Kilowattstunde gewonnen aus Wasserstoff 2003 noch mindestens 5000 Euro kostete.
Für den Ottomotor ist dies wesentlich billiger mit rund 50 Euro pro Kilowattstunde.
Dies ist momentan noch eines der größten Probleme, die es zu bewältigen gibt.
Die Brennstoffzelle ist umweltfreundlich, in großen Massen zu vermarkten, kommt jedoch auf wesentlich zu hohe Kosten.
Die Wasserstoffzelle muss wesentlich günstiger werden und dies geht nur mit Forschung.
Eine andere möglichkeit, jedoch nicht genügend mobil, stellt die Solarzelle da.
Auch hier ist Verschleiß und Nutzen noch teuer.
Die Solarzellen im Vertrieb besitzen noch einen Wirkungsgrad von unter 20\%.
Hier ist aktuell noch die Kernkraft vorherrschender Energielieferant. Doch nach zwischenfällen wie Fukushima und dem Endlagerproblem, kann dies auch nicht die Lösung bleiben.
Doch auch hier ist in der Forschung die Lösung, denn im Labor wurden schon Zellen mit einem Wirkungsgrad von 50\% hergestellt.
Als zukünftige Generation liegt es nun an uns, sich mit diesem Thema einhergehend zu Beschäftigen um Lösungen für die Zukunft finden zu können.
Hierzu Befassen wir uns im Kommenden mit diesen Beiden Zellen und gehen auf die Wirkungsprinzipien ein.
Nachhaltige Energiequellen sind notwendig für das Bestehen unserer Fortschrittlichen Gesellschaft.
\begin{center}
"Der hundertprozentige Umstieg auf erneuerbare Energien ist nicht nur ökologisch, sondern auch ökonomisch geboten."\\
Franz Alt in Sonnige Aussichten, 2008 
\end{center}
