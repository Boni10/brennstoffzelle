Mit immer größer werdender Kritik an fossilen Brennstoffen in Motorbetriebenen Fahrzeugen und Flugzeugen wird die Suche nach einem passenden Ersatz immer bedeutender. Nachdem sich nun auch das Elektroauto aufgrund seines $CO_2$ Abdrucks große Kritik einholte, ist eine Andere Technologie notwendig. Eine Technologie, die Effizient, umweltschonend und in großen Massen herstellbar ist.
Über eine mögliche Alternative soll nun hier berichtet werden, die Brennstoff- und Solarzelle.
Ottomotoren haben einen geringen Wirkungsgrad und nutzen den endlichen fossilen Brennstoff Erdöl.
Rund 15 \% der Erde bestehen aus Wasserstoff, der sich zu seiner elementaren Form auch leicht herstellen lässt.
Die Brennstoffzelle scheint also als die optimale Wahl zu sein.
Das größte bisherige Problem ist, dass erst Wasserstoff durch Strom hergestellt werden muss, dann kann er nur gefährlich (da Hoch explosiv) gelagert und später nur unter einem weiteren Verlust zu Strom rücktransformiert werden.
Für den Ottomotor ist dies wesentlich billiger, da man lediglich Erdöl abbauen und zu Benzin/Disel umwandeln muss.
Eine der größten Herausforderungen momentan ist also, die Effizienz der Brennstoffzelle und die Lagerung des Wasserstoffs so weit zu optimieren, dass die Zelle ein würdiger Ersatz zum Ottomotor oder dem Elektroauto werden kann.
Die Brennstoffzelle ist umweltfreundlich, in großen Massen zu vermarkten, kommt jedoch auf wesentlich zu hohe Kosten.
Zur Optimierung muss nun also Forschung betrieben werden.

Eine andere möglichkeit, jedoch nicht genügend mobil für Fahrzeuge, stellt die Solarzelle da.
Hier ist Verschleiß und Nutzen noch sehr teuer im Vergleich zu Atom- und Kohlekraftwerken.
Die Solarzellen im handelsüblichen Vertrieb besitzen noch einen Wirkungsgrad von unter 20\%.
Dies ist deutlich zu wenig, wenn man überlegt, dass im Labor schon Wirkungsgrade von knapp 50\% üblich sind. 
Nach zwischenfällen wie Fukushima oder Tschernobyl und dem Problem der Endlagerung, wäre es durchaus sinnvoll, in Nachhaltigkeit zu investieren, anstatt in Schadensbegrenzung. Diese Zwischenfälle haben der Atomkraft schon mehrere Billionen Euro schaden zugeschrieben.
Forschung für kostengünstige und effiziente Solarzellen wäre also eine sehr gute Option.
Als zukünftige Generation liegt es nun an uns, sich mit diesem Thema einhergehend zu Beschäftigen um Lösungen für die Zukunft zu  finden.

Hierzu Befassen wir uns im Kommenden mit diesen Beiden Zellen und gehen auf die Wirkungsprinzipien ein.
Nachhaltige Energiequellen sind notwendig für das Bestehen unserer Fortschrittlichen Gesellschaft.
Hier noch ein Zitat zur Anregung, wie weit die Wirtschaft schön wäre.
\begin{center}
"Der hundertprozentige Umstieg auf erneuerbare Energien ist nicht nur ökologisch, sondern auch ökonomisch geboten."\\
Franz Alt in Sonnige Aussichten, 2008 
\end{center}
